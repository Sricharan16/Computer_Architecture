\documentclass[12pt]{article}
%	options include 12pt or 11pt or 10pt
%	classes include article, report, book, letter, thesis

\title{CS2323 Computer Architecture}
\author{V Sri Charan Reddy \\ CS16BTECH11044}
\date{19 September 2016}

\begin{document}
\maketitle

This document is generated by \LaTeX

\section{}
Given address = 0x CDDBAA\\
Its representation in binary form = 0b 1100 1101 1101 1011 1010 1010\\
Cache size = No.of.blocks * Block size\\
           = No of sets* No of ways * blocksize\\
Given, no of sets = 4096\\
no of ways =16\\
block size = 64Bytes\\
Cache size = 4MB\\
block offset = $log_2 64 = 6$ bits\\ 
set index= $log_2 4096 = 12$\\
The LSB starts with the blockoffset then the set index \\


	\begin{tabular}{ |c|c|c| } 
		\hline
		Tag & Set Index & Offset \\ 
		1100 11 & 01 1101 1011 10 & 10 1010\\  
		\hline
	\end{tabular}\\
Its set index is 01 1101 1011 10 in binary \\
In base 10, set index = 1902\\
It will be mapped to any of the 16 ways depending if the place is occupied or not. The default assigning is to the first way that is available, if occupied the replacement tehniques are used and the used blocks are replaced.\\


\section{}
Given , \\
No of accesses per second = 50000000\\
leakage power = 0.08W\\
dynamic energy = 0.9nJ/access\\
In one second, leakage energy= leakage power*time\\
leakage energy = 0.08*1 =0.08J\\
dynamic energy = $.9 * 10^{-9} *50000000$ =0.045J\\
leakage fraction =
\(\frac{leakage energy}{leakage energy + dynamic energy}\) \\
leakage fraction =
\(\frac{0.08}{0.08+0.045}\)\      =
0.64\\
leakage percentage = leakage * 100\\
leakage = 64 percent \\  
\section{}
Given block size =4\\
offset = $log_2 4 = 2$ bits\\ 
8 sets means set index is $log_2 8 =3$ bits\\


\begin{center}
	\begin{tabular}{ |c|c|c|c| } 
		\hline
		  & Tag & Set Index  & Offset \\
		  \hline
		0 & 000 & 000 & 00\\
		63&001&111&11\\
	   	1&000&000&01\\
	   	62&001&111&10\\
	   	2&000&000&10\\
	   	61&001&111&01\\
	   	3&000&000&11\\
	   	60&001&111&00\\
	   	4&000&001&00\\
	   	59&001&110&11 \\
	   	5 &000&001&01\\
	   	58 &001&110&10\\
	   	6 &000&001&10\\
	   	57 &001&110&01\\
	   	7 &000&001&11\\
	   	56 &001&110&00\\
	   	8 &000&010&00\\
	   	55 &001&101&11\\
	   	9 &000&010&01\\
	   	54 &001&101&10\\
	   	10&000&010&10\\
	   	53&001&101&01\\
	   	11 &000&010&11\\
	   	52&001&101&00\\
		\hline
	\end{tabular}
\end{center}
 cache-1 \\ 
Tag is the first 3 bits and set index is the next 3 bits and the offset is in next 2 bits.\\


\begin{center}
	\begin{tabular}{ |c|c|c| } 
		\hline
		3bits & 3 bits & 2 bits \\ 
		\hline
		Tag & Setindex & offset \\  
		\hline
	\end{tabular}
\end{center}

cache-2\\
Tag is the last 3 bits, and set index is the before 3 bits and the rest is offset.
\begin{center}
	\begin{tabular}{ |c|c|c| } 
		\hline
		2 bits & 3 bits & 3 bits \\ 
		\hline
	    Offset & Setindex & Tag \\  
		\hline
	\end{tabular}
\end{center}
a)For accessing the sequence by cache 1 
After comparing set index and tag the hit miss sequence of these address access is:\\


\begin{center}
	\begin{tabular}{ |c|c|c|c|c|c|c|c|c|c|c|c|c|c|c|c|c|c|c|c|c|c|c|c|c| } 
	\hline
		0 & 63 & 1& 62 & 2 & 61 & 3 & 60 & 4 & 59 & 5 & 58 & 6 & 57 & 7 & 56 & 8 & 55  & 9   \\ 
		
		M&M&H&H&H&H&H&H&M&M&H&H&H&H&H&H&M&M&H \\  
		\hline
	\end{tabular}
\end{center}


	\begin{tabular}{ |c|c|c| c|c|} 
		\hline
		 54 & 10 & 53 & 11 & 52 \\ 
		 
		H&H&H&H&H \\ 
	 
		\hline
	\end{tabular}\\
\\
ratio  =
\(\frac{hits}{access}\) =
\(\frac{18}{24}\)\      =
\(\frac{3}{4}\)\     \\

For accessing the sequence by cache 2
After comparing set index and tag the hit miss sequence of these address access is:\\


\begin{center}
	\begin{tabular}{ |c|c|c|c|c|c|c|c|c|c|c|c|c|c|c|c|c|c|c|c|c|c|c|c|c| } 
		\hline
		0 & 63 & 1& 62 & 2 & 61 & 3 & 60 & 4 & 59 & 5 & 58 & 6 & 57 & 7 & 56 & 8 & 55  & 9   \\ 
		
		M&M&M&M&M&M&M&M&M&M&M&M&M&M&M&M&M&M&M \\  
		\hline
	\end{tabular}
\end{center}


\begin{tabular}{ |c|c|c| c|c|} 
	\hline
	54 & 10 & 53 & 11 & 52 \\ 
	
	M&M&M&M&M \\ 
	
	\hline
\end{tabular}\\
\\
ratio  =
\(\frac{hits}{access}\) =
\(\frac{0}{24}\)\      =
0     \\
\\
b)

\begin{center}
	\begin{tabular}{ |c|c|c|c| } 
		\hline
		& Tag & Setindex & offset \\ 
		\hline
		0&000 & 000 & 00 \\ 
		64&010&000&00\\ 
		128&100&000&00\\
		192&110&000&00\\
		1&000&000&01\\
		65&010&000&01\\
		129&100&000&01\\
		193&110&000&01\\
		11&000&010&11\\
		75&010&010&11\\
		139&100&010&11\\
		203&110&010&11\\
		9&000&010&01\\
		137&100&010&01\\
		201&110&010&01\\
		73&010&010&01\\
		\hline
	\end{tabular}
\end{center}
For accessing the sequence by cache 1
After comparing set index and tag the hit miss sequence of these address access is:\\
\begin{center}
	\begin{tabular}{ |c|c|c|c|c|c|c|c|c|c|c|c|c|c|c|c| } 
		\hline
		0&64&128&192&1&65&129&193&11&75&139&203&9&137&201&73 \\ 
		
		M&M&M&M&M&M&M&M&M&M&M&M&M&M&M&M \\  
		\hline
	\end{tabular}
\end{center}
ratio  =
\(\frac{hits}{access}\) =
\(\frac{0}{16}\)\      =
0     \\
For accessing the sequence by cache 2
After comparing set index and tag the hit miss sequence of these address access is:\\
\begin{center}
	\begin{tabular}{ |c|c|c|c|c|c|c|c|c|c|c|c|c|c|c|c| } 
		\hline
		0&64&128&192&1&65&129&193&11&75&139&203&9&137&201&73 \\ 
		
		M&H&H&H&M&H&H&H&M&H&H&H&M&H&H&H \\  
		\hline
	\end{tabular}
\end{center}
ratio  =
\(\frac{hits}{access}\) =
\(\frac{12}{16}\)\      =
\(\frac{3}{4}\)\    \\\\
c)\\
In part a cache-1 is better than cache-2 as the ratio of hits:access is higher for cache-1\\
In part b cache-2 is better than cache-1 as the ratio of hits:access is higher for cache-2\\
As the hits:access changes , a particular kind of cache is not always better\\

\section{}
Frequency of P1 = 2 GHz\\
Frequency of P2 = 1.5 GHz \\
No.of instruction in program = $10^{6}$\\\\
No of instructions in A
=\(\frac{20}{100}\) $*10^{6}$  = 2 * $10^{5}$\\\\
No of instructions in B
=\(\frac{25}{100}\) $*10^{6}$  = 25 * $10^{4}$\\\\
No of instructions in C
=\(\frac{40}{100}\) $*10^{6}$  = 4 * $10^{5}$\\\\
No of instructions in D
=\(\frac{15}{100}\) $*10^{6}$  = 15 * $10^{4}$\\\\
CPI of P1 A=1,B=2,C=3,D=4 \\
Total cycles for P1 to run a program = $\sum_{i=1}^{\i=4}
 A_{i}$ * instructions in $A_{i}$\\
 = 1*2 * $10^{5}$ + 2 *25 * $10^{4}$ + 3 *4 * $10^{5}$ + 4 *15 * $10^{4}$\\ 
 =$25 * 10^{5}$\\
Time taken for P1 to run a program = 
\(\frac{no. of .cycles}{frequency}\)\\=
\(\frac{25 * 10^{5}}{2GHz}\)\\
CPI of P2 A=2,B=2,C=2,D=2 \\
Total cycles for P1 to run a program 
 = 2*2 * $10^{5}$ + 2 *25 * $10^{4}$ + 2 *4 * $10^{5}$ + 2 *15 * $10^{4}$\\
 = 2 * $ 10^{6}$\\
 Time taken for P2 to run a program = 
 \(\frac{no. of .cycles}{frequency}\)\\=
 \(\frac{2 * 10^{6}}{1.5GHz}\)\\
As time taken for P1 is less than P2 \\
So P1 is faster for this program
\section{}
After every write miss P+1 bit will be 1 and after every read miss P+1 bit will be 0\\


	\begin{tabular}{ |c|c|c|c|c| } 
		
		P0 & P1 & P2 & P3 & P+1 \\ 
		\hline
		1 & 0 & 0 & 0 & 0 \\ 
		\hline
	\end{tabular}
After P0 has read miss\\


\begin{tabular}{ |c|c|c|c|c| } 
	
	P0 & P1 & P2 & P3 & P+1 \\ 
	\hline
	0 & 1 & 0 & 0 & 1 \\ 
	\hline
\end{tabular}
After P1 has write miss\\


\begin{tabular}{ |c|c|c|c|c| } 
	
	P0 & P1 & P2 & P3 & P+1 \\ 
	\hline
	0 & 0 & 1 & 0 & 1 \\ 
	\hline
\end{tabular}
After P2 has write miss\\


\begin{tabular}{ |c|c|c|c|c| } 
	
	P0 & P1 & P2 & P3 & P+1 \\ 
	\hline
	0 & 0 & 1 & 1 & 0 \\ 
	\hline
\end{tabular}
After P3 has read miss\\


\begin{tabular}{ |c|c|c|c|c| } 
	
	P0 & P1 & P2 & P3 & P+1 \\ 
	\hline
	0 & 0 & 0 & 1 & 1 \\ 
	\hline
\end{tabular}
After P3 has write miss\\

\begin{tabular}{ |c|c|c|c|c| } 
	
	P0 & P1 & P2 & P3 & P+1 \\ 
	\hline
	0 & 1 & 0 & 1 & 0 \\ 
	\hline
\end{tabular}
After P1 has read miss\\


\section{}
Two applictions are running on L2 cache \
Given that number of misses are scaled linearly \\
So difference of misses between successive ways for application 1\\
=\(\frac{1000-500}{6-2}\) = 125\\
So difference of misses between successive ways for application 2 \\
=\(\frac{2000-1800}{6-2}\) = 50\\


\begin{center}
	\begin{tabular}{ |c|c|c|c|c|c| } 
		\hline
		way & 2 & 3 & 4 & 5 & 6 \\ 
		\hline
		application1  & 1000 & 875 &750 &625 & 500 \\ 
		application2 & 2000 & 1950 & 1900 & 1850 & 1800 \\ 
		\hline
	\end{tabular}
\end{center}
As the cache is given as 8 way,\\
To improve the performance we must have less misses \\
misses =  ways in application1 * misses in application1 +ways in application2 *misses in application2 


\begin{center}
	\begin{tabular}{ |c|c|c| } 
		\hline
		Application1(ways) & Application2(ways) & misses \\ 
		\hline
		2 & 6 & 2*1000+6*1800 =12800\\ 
		3 & 5 & 3* 875+5*1850 =11875\\ 
		4 & 4 & 4*750+4*1900 =10600\\
		5 & 3 & 5*625+3*1950 =8975\\
		6 & 2 & 6*500+2*2000 =7000 \\
		\hline
	\end{tabular}
\end{center}
So misses are minimum if 6 ways for application1 and 2 ways for application2\\




\end{document}