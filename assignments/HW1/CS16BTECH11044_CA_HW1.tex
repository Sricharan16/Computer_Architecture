% --------------------------------------------------------------
% This is all preamble stuff that you don't have to worry about.
% Head down to where it says "Start here"
% --------------------------------------------------------------
 
\documentclass[12pt]{article}
 
\usepackage[margin=1in]{geometry} 
\usepackage{amsmath,amsthm,amssymb}
\newcommand{\N}{\mathbb{N}}
\newcommand{\Z}{\mathbb{Z}}
 
\newenvironment{theorem}[2][Theorem]{\begin{trivlist}
\item[\hskip \labelsep {\bfseries #1}\hskip \labelsep {\bfseries #2.}]}{\end{trivlist}}
\newenvironment{lemma}[2][Lemma]{\begin{trivlist}
\item[\hskip \labelsep {\bfseries #1}\hskip \labelsep {\bfseries #2.}]}{\end{trivlist}}
\newenvironment{exercise}[2][Exercise]{\begin{trivlist}
\item[\hskip \labelsep {\bfseries #1}\hskip \labelsep {\bfseries #2.}]}{\end{trivlist}}
\newenvironment{problem}[2][Problem]{\begin{trivlist}
\item[\hskip \labelsep {\bfseries #1}\hskip \labelsep {\bfseries #2.}]}{\end{trivlist}}
\newenvironment{question}[2][Question]{\begin{trivlist}
\item[\hskip \labelsep {\bfseries #1}\hskip \labelsep {\bfseries #2.}]}{\end{trivlist}}
\newenvironment{corollary}[2][Corollary]{\begin{trivlist}
\item[\hskip \labelsep {\bfseries #1}\hskip \labelsep {\bfseries #2.}]}{\end{trivlist}}
 
\begin{document}
 
% --------------------------------------------------------------
%                         Start here
% --------------------------------------------------------------
 
\title{Homework 1}%replace X with the appropriate number
\author{V Sri Charan Reddy\\ %replace with your name
CS2323 Computer Architecture(CS16BTECH11044)} %if necessary, replace with your course title
\maketitle
	This document is generated by \LaTeX
\begin{enumerate}

	\item      Given No of accesses = 50000000 accesses/sec\\
	          Leakage power = 0.07W\\
              Dynamic Energy =0.8nJ/access \\
	          Execution time = 1sec\\
              leakage energy = 0.07 * 1 J = 0.07J\\
	          Dynamic energy = 0.8nJ * 50000000 = 0.8*0.05 = 0.04J\\
				\% of leak energy =(0.07)/(0.07+0.04)
				=63.63\%.\\
	\item TLB is amount of memory accessible from the TLB.\\
	TLB Reach=(TLB Size)*(Page Size)\\
	TLB1 Reach= (4kB)*(64) = 256kB\\
	TLB2 Reach= (2MB)*(32) = 64MB \\
	TLB3 Reach= (1GB)*(8)  = 8GB \\
	TLB Reach = 256kB+64MB+8GB = 8.064256GB\\
	\item Given block size =4\\
	offset = $log_2 4 = 2$ bits\\ 
	8 sets means set index is $log_2 8 =3$ bits\\
	
	
	\begin{center}
		\begin{tabular}{ |c|c|c|c| } 
			\hline
			& Tag & Set Index  & Offset \\
			\hline
			0 & 000 & 000 & 00\\
			63&001&111&11\\
			1&000&000&01\\
			62&001&111&10\\
			2&000&000&10\\
			61&001&111&01\\
			3&000&000&11\\
			60&001&111&00\\
			4&000&001&00\\
			59&001&110&11 \\
			5 &000&001&01\\
			58 &001&110&10\\
			6 &000&001&10\\
			57 &001&110&01\\
			7 &000&001&11\\
			56 &001&110&00\\
			8 &000&010&00\\
			55 &001&101&11\\
			9 &000&010&01\\
			54 &001&101&10\\
			10&000&010&10\\
			53&001&101&01\\
			11 &000&010&11\\
			52&001&101&00\\
			\hline
		\end{tabular}
	\end{center}
	cache-1 \\ 
	Tag is the first 3 bits and set index is the next 3 bits and the offset is in next 2 bits.\\
	
	
	\begin{center}
		\begin{tabular}{ |c|c|c| } 
			\hline
			3bits & 3 bits & 2 bits \\ 
			\hline
			Tag & Setindex & offset \\  
			\hline
		\end{tabular}
	\end{center}
	
	cache-2\\
	Tag is the last 3 bits, and set index is the before 3 bits and the rest is offset.
	\begin{center}
		\begin{tabular}{ |c|c|c| } 
			\hline
			2 bits & 3 bits & 3 bits \\ 
			\hline
			Offset & Setindex & Tag \\  
			\hline
		\end{tabular}
	\end{center}
	a)For accessing the sequence by cache 1 
	After comparing set index and tag the hit miss sequence of these address access is:\\
	
	
	\begin{center}
		\begin{tabular}{ |c|c|c|c|c|c|c|c|c|c|c|c|c|c|c|c|c|c|c|c|c|c|c|c|c| } 
			\hline
			0 & 63 & 1& 62 & 2 & 61 & 3 & 60 & 4 & 59 & 5 & 58 & 6 & 57 & 7 & 56 & 8 & 55  & 9   \\ 
			
			M&M&H&H&H&H&H&H&M&M&H&H&H&H&H&H&M&M&H \\  
			\hline
		\end{tabular}
	\end{center}
	
	
	\begin{tabular}{ |c|c|c| c|c|} 
		\hline
		54 & 10 & 53 & 11 & 52 \\ 
		
		H&H&H&H&H \\ 
		
		\hline
	\end{tabular}\\
	\\
	ratio  =
	\(\frac{hits}{access}\) =
	\(\frac{18}{24}\)\      =
	\(\frac{3}{4}\)\     \\
	
	For accessing the sequence by cache 2
	After comparing set index and tag the hit miss sequence of these address access is:\\
	
	
	\begin{center}
		\begin{tabular}{ |c|c|c|c|c|c|c|c|c|c|c|c|c|c|c|c|c|c|c|c|c|c|c|c|c| } 
			\hline
			0 & 63 & 1& 62 & 2 & 61 & 3 & 60 & 4 & 59 & 5 & 58 & 6 & 57 & 7 & 56 & 8 & 55  & 9   \\ 
			
			M&M&M&M&M&M&M&M&M&M&M&M&M&M&M&M&M&M&M \\  
			\hline
		\end{tabular}
	\end{center}
	
	
	\begin{tabular}{ |c|c|c| c|c|} 
		\hline
		54 & 10 & 53 & 11 & 52 \\ 
		
		M&M&M&M&M \\ 
		
		\hline
	\end{tabular}\\
	\\
	ratio  =
	\(\frac{hits}{access}\) =
	\(\frac{0}{24}\)\      =
	0     \\
	\\
	b)
	
	\begin{center}
		\begin{tabular}{ |c|c|c|c| } 
			\hline
			& Tag & Setindex & offset \\ 
			\hline
			0&000 & 000 & 00 \\ 
			64&010&000&00\\ 
			128&100&000&00\\
			192&110&000&00\\
			1&000&000&01\\
			65&010&000&01\\
			129&100&000&01\\
			193&110&000&01\\
			11&000&010&11\\
			75&010&010&11\\
			139&100&010&11\\
			203&110&010&11\\
			9&000&010&01\\
			137&100&010&01\\
			201&110&010&01\\
			73&010&010&01\\
			\hline
		\end{tabular}
	\end{center}
	For accessing the sequence by cache 1
	After comparing set index and tag the hit miss sequence of these address access is:\\
	\begin{center}
		\begin{tabular}{ |c|c|c|c|c|c|c|c|c|c|c|c|c|c|c|c| } 
			\hline
			0&64&128&192&1&65&129&193&11&75&139&203&9&137&201&73 \\ 
			
			M&M&M&M&M&M&M&M&M&M&M&M&M&M&M&M \\  
			\hline
		\end{tabular}
	\end{center}
	ratio  =
	\(\frac{hits}{access}\) =
	\(\frac{0}{16}\)\      =
	0     \\
	For accessing the sequence by cache 2
	After comparing set index and tag the hit miss sequence of these address access is:\\
	\begin{center}
		\begin{tabular}{ |c|c|c|c|c|c|c|c|c|c|c|c|c|c|c|c| } 
			\hline
			0&64&128&192&1&65&129&193&11&75&139&203&9&137&201&73 \\ 
			
			M&H&H&H&M&H&H&H&M&H&H&H&M&H&H&H \\  
			\hline
		\end{tabular}
	\end{center}
	ratio  =
	\(\frac{hits}{access}\) =
	\(\frac{12}{16}\)\      =
	\(\frac{3}{4}\)\    \\\\
	\item Frequency of P1 = 2.2 GHz\\
	Frequency of P2 = 1.6 GHz \\
	No.of instruction in program = $10^{6}$\\\\
	No of instructions in A
	=\(\frac{20}{100}\) $*10^{6}$  = 2 * $10^{5}$\\\\
	No of instructions in B
	=\(\frac{25}{100}\) $*10^{6}$  = 25 * $10^{4}$\\\\
	No of instructions in C
	=\(\frac{45}{100}\) $*10^{6}$  = 45 * $10^{4}$\\\\
	No of instructions in D
	=\(\frac{10}{100}\) $*10^{6}$  = 1 * $10^{5}$\\\\
	CPI of P1 A=1,B=2,C=3,D=4 \\
	Total cycles for P1 to run a program = $\sum_{i=1}^{\i=4}
	A_{i}$ * instructions in $A_{i}$\\
	= 1*2 * $10^{5}$ + 2 *25 * $10^{4}$ + 3 *45 * $10^{4}$ + 4 *1 * $10^{5}$\\ 
	=$24.5 * 10^{5}$\\
	Time taken for P1 to run a program = 
	\(\frac{no. of .cycles}{frequency}\)\\=
	\(\frac{24.5 * 10^{5}}{2.2GHz}\)\\
	CPI of P2 A=2,B=2,C=2,D=2 \\
	Total cycles for P2 to run a program 
	= 2*2 * $10^{5}$ + 2 *25 * $10^{4}$ + 2 *45 * $10^{4}$ + 2 *1 * $10^{5}$\\
	= 2 * $ 10^{6}$\\
	Time taken for P2 to run a program = 
	\(\frac{no. of .cycles}{frequency}\)\\=
	\(\frac{2 * 10^{6}}{1.6GHz}\)\\
	As time taken for P1 is less than P2 \\
	So P1 is faster for this program.\\
	\item After every write miss P+1 bit will be 1 and after every read miss P+1 bit will be 0\\
	
	
	\begin{tabular}{ |c|c|c|c|c| } 
		
		P0 & P1 & P2 & P3 & P+1 \\ 
		\hline
		1 & 0 & 0 & 0 & 0 \\ 
		\hline
	\end{tabular}
	After P0 has read miss\\
	
	
	\begin{tabular}{ |c|c|c|c|c| } 
		
		P0 & P1 & P2 & P3 & P+1 \\ 
		\hline
		0 & 1 & 0 & 0 & 1 \\ 
		\hline
	\end{tabular}
	After P1 has write miss\\
	
	
	\begin{tabular}{ |c|c|c|c|c| } 
		
		P0 & P1 & P2 & P3 & P+1 \\ 
		\hline
		0 & 0 & 0 & 1 & 1 \\ 
		\hline
	\end{tabular}
	After P3 has write miss\\
	
	
	\begin{tabular}{ |c|c|c|c|c| } 
		
		P0 & P1 & P2 & P3 & P+1 \\ 
		\hline
		0 & 0 & 1 & 1 & 0 \\ 
		\hline
	\end{tabular}
	After P2 has read miss\\
	
	
	\begin{tabular}{ |c|c|c|c|c| } 
		
		P0 & P1 & P2 & P3 & P+1 \\ 
		\hline
		0 & 0 & 1 & 0 & 1 \\ 
		\hline
	\end{tabular}
	After P2 has write miss\\
	
	\begin{tabular}{ |c|c|c|c|c| } 
		
		P0 & P1 & P2 & P3 & P+1 \\ 
		\hline
		1 & 0 & 1 & 0 & 0 \\ 
		\hline
	\end{tabular}
	After P0 has read miss\\
	
	\item Two applictions are running on L2 cache \
	Given that number of misses are scaled linearly \\
	So difference of misses between successive ways for application 1\\
	=\(\frac{1000-400}{6-2}\) = 150\\
	So difference of misses between successive ways for application 2 \\
	=\(\frac{2000-1800}{6-2}\) = 50\\
	
	
	\begin{center}
		\begin{tabular}{ |c|c|c|c|c|c| } 
			\hline
			way & 2 & 3 & 4 & 5 & 6 \\ 
			\hline
			application1  & 1000 & 850 &700 & 550 & 400 \\ 
			application2 & 2000 & 1950 & 1900 & 1850 & 1800 \\ 
			\hline
		\end{tabular}
	\end{center}
	As the cache is given as 8 way,\\
	To improve the performance we must have less misses \\
	misses =  ways in application1 * misses in application1 +ways in application2 *misses in application2 
	
	
	\begin{center}
		\begin{tabular}{ |c|c|c| } 
			\hline
			Application1(ways) & Application2(ways) & misses \\ 
			\hline
			2 & 6 & 2*1000+6*1800 =12800\\ 
			3 & 5 & 3* 850+5*1850 =11800\\ 
			4 & 4 & 4*700+4*1900 =10400\\
			5 & 3 & 5*550+3*1950 =8600\\
			6 & 2 & 6*400+2*2000 =6400 \\
			\hline
		\end{tabular}
	\end{center}
	So misses are minimum if 6 ways for application1 and 2 ways for application2\\
\item Given transaction rate of P=34/min\\
transaction rate of Q=58/min\\
transaction rate of R=81/min
Given each of them make 500 transactions. \\
Total no of transactions=500*3=1500\\
Time taken for running 1500 transactions=(500/34)+(500/58)+(500/81)\\
= 29.49min\\
Average transactions per minute =(1500/29.49)=50.86 transactions/min\\
We use harmonic mean here.\\
\item 
Assume 100s of time period. 
Time taken by the system0 is 100sec.(since it is single core).\\
time taken by system1 for initialisation =25sec.\\
time taken by system1 for vision-processing =37sec.\\
time taken by system1 for signal-processing =38sec.\\
Overall time for system1=25+(38/10)+(37/6).\\
=34.96\\
Speedup of system1 over system0 is (100/34.96) i.e 2.86\\
\item 
	Given that running at 3GHz and 1Volt the program takes 30 sec to execute. Voltage is linearly dependent on the frequency. If voltage is 1.2V then frequency becomes 1.2 times.\\
	(i)Hence the smallest time it needs to execute a program is (30/1.2)sec=25sec.\\
	(ii)Dynamic power=V*V*f=V*V*V(since V and f are linearly related)\\
	Leakage power is also proportional to V\\
	Total power =(80*V*V*V)+30*V\\
				=64.96W.\\
	(iii)Energy = power*time\\
    	=((80*V*V*V)+30*V)*(30/V)\\
    	=((80*V*V*V)+30)*(30/V)\\
    	=((80*V*V)+(30))*30\\
    	=2436J
\item 
Given virtual address =48bit\\
Physical memory =2GB=31 bits addressing\\
Page size is given as 2kB means offset=11bits.\\
The offset length remains same in virtual as well as the physical address.\\
Number of bytes for storing one map in TLB = 48+31-11-11 =57bits \\
Number of entries =32.\\
The size of TLB =32*57 bits= 228Bytes.\\
\item 
In a given cycle, the addresses coming to the four ports of TLB are:\\
0x4795BA21, 0x4795BB21, 0x5795BA21 and 0x4785BA21.\\
Frame size is given as 1kB,offset=10\\
Here the address 0x4795BA21,0x4795BB21 are same if we do not consider offset.That means both are same.\\
So no of unique accesses =4-1=3.\\
\item
(a)By following the LRU policy, we get the following hit(H)/miss(M) sequence: \\
\begin{center}
	\begin{tabular}{ |c|c|c|c|c|c|c|c|c|c|c|c| } 
		\hline
		A & B & C & D & A & B & C & D & A & B & C & D\\ 
		\hline
		M & M & M & M & M & M & M & M & M & M & M & M\\ 
		\hline
	\end{tabular}
\end{center}
(b)By following the MRU policy, we get the following hit(H)/miss(M) sequence:\\
\begin{center}
	\begin{tabular}{ |c|c|c|c|c|c|c|c|c|c|c|c| } 
		\hline
		A & B & C & D & A & B & C & D & A & B & C & D\\ 
		\hline
		M & M & M & M & H & M & M & H & M & M & H & M\\ 
		\hline
	\end{tabular}
\end{center}
	
\end{enumerate}
\end{document}